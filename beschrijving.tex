\documentclass{article}
\usepackage{amsmath, amssymb}

\begin{document}
	
	\section*{Problem Description}
	
	We aim to optimize the allocation of resources across multiple projects. Specifically, we assign individuals to projects while determining their level of involvement, maximizing the total subsidy earnings.

	
	\subsection*{Decision Variables}
	Let \( x_i^j \in [0,1] \) be a continuous decision variable, where:
	\begin{itemize}
		\item \( i \) represents a project.
		\item \( j \) represents a person.
		\item \( x_i^j \) denotes the hours person \( j \) dedicates to project \( i \).
	\end{itemize}
	
	\subsection*{Objective Function}
	The objective is to maximize the total subsidy earnings, defined as:
	\[
	\max \sum_{i} \sum_{j} \alpha_j  \cdot P_j\cdot x_i^j
	\]
	where:
	\begin{itemize}
		\item \( P_j \) is the cost per person $j$ expressed as rate per hour.
		\item \( \alpha_j \) is a weighting factor accounting for the subsidy percentage.
	\end{itemize}
	
	\subsection*{Constraints}
	\begin{enumerate}
		\item \textbf{Project Resource Requirement:  }
		Each project requires a fixed level of involvement:
		\[
		\sum_j x_i^j  = \delta_i, \quad \forall i
		\]
		where \( \delta_i \) is the required man months for project \( i \).
	
	    \item \textbf{Individual Workload Limit:  }
		Each person can only work up to a full-time equivalent (FTE) of 1720 per year:
		\[
		\sum_i x_i^j \leq 1720, \quad \forall j
		\]
		
		\item \textbf{Project-Specific Constraints:  }
		Certain projects may have individual-specific constraints restricting some people or forcing people from participating:
		\[
		x_i^j = 0, \quad \forall i, j
		\]
		\[
		x_i^j >= \phi_i^j, \quad \forall i, j
		\]
		
\end{enumerate}

\section*{Additional Constraint: Minimum Hours per Profile/Role}

For each project \( i \) and for every required profile (or role) \( r \), we impose that the total number of allocated hours by all individuals possessing that profile must be at least the specified threshold \( H_{i,r} \) (in hours) as given in the \texttt{project\_roles.csv} file. This constraint ensures that each project receives the necessary expertise and effort from the appropriate profiles.

Mathematically, the constraint is formulated as:
\[
\sum_{j \in J_r} x_i^j \ge H_{i,r} \quad \forall i \in I,\; \forall r \in R_i
\]
where:
\begin{itemize}
	\item \( I \) denotes the set of all projects.
	\item \( R_i \) denotes the set of required roles for project \( i \).
	\item \( J_r \) denotes the set of individuals whose profile matches role \( r \).
	\item \( x_i^j \) is the number of hours that individual \( j \) is allocated to project \( i \).
	\item \( H_{i,r} \) is the minimum number of hours required for role \( r \) on project \( i \), as specified in the project roles file.
\end{itemize}

This constraint guarantees that, for every project, the cumulative contribution (in hours) from resources with the necessary profile meets or exceeds the project's minimum requirement.


\subsection*{Solution Approach}
This problem is formulated as a linear programming (LP) optimization problem and can be (hopefully) efficiently solved using the Branch and Bound method.

	
\end{document}
